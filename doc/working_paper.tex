\documentclass[11pt,a4paper]{article}

% Packages
\usepackage[utf8]{inputenc}
\usepackage[T1]{fontenc}
\usepackage{geometry}
\usepackage{amsmath}
\usepackage{amsfonts}
\usepackage{amssymb}
\usepackage{graphicx}
\usepackage{hyperref}
\usepackage{cite}
\usepackage{enumitem}
\usepackage{booktabs}
\usepackage{array}
\usepackage{longtable}
\usepackage{float}
\usepackage{wrapfig}
\usepackage{rotating}
\usepackage{textcomp}
\usepackage{marvosym}
\usepackage{wasysym}
\usepackage{amsthm}
\usepackage{thmtools}
\usepackage{listings}
\usepackage{fancyhdr}
\usepackage{parskip}
\usepackage{setspace}
\usepackage{xcolor}
\usepackage{titlesec}
\usepackage{tcolorbox}

\usepackage{listings}

\usepackage{framed}
\usepackage{tcolorbox}
\usepackage{fancybox}
\usepackage{mdframed}
\usepackage{adjustbox}
\usepackage{subcaption}

\fontsize{6}{8}\selectfont

% Page setup
\geometry{margin=1in}
\pagestyle{fancy}
\fancyhf{}
\rhead{Quantum Cognition Machine Learning (QCML)}
\lhead{Technical Documentation}
\rfoot{\thepage}

% Fix headheight warning
\setlength{\headheight}{14pt}

% Title formatting
\titleformat{\section}{\Large\bfseries}{\thesection}{1em}{}
\titleformat{\subsection}{\large\bfseries}{\thesubsection}{1em}{}
\titleformat{\subsubsection}{\normalsize\bfseries}{\thesubsubsection}{1em}{}

% Hyperref setup
\hypersetup{
    colorlinks=true,
    linkcolor=blue,
    filecolor=magenta,      
    urlcolor=cyan,
    citecolor=red,
    pdftitle={Quantum Cognition Machine Learning: Technical Documentation},
    pdfauthor={QCML Research Team},
    pdfsubject={Quantum Cognition Machine Learning and Optimization},
    pdfkeywords={quantum cognition, machine learning, manifold learning, dimension estimation, optimization}
}

% Bibliography setup
\bibliographystyle{ieeetr}

% Document begin
\begin{document}

% Title page
\begin{titlepage}
    \centering
    \vspace*{2cm}
    
    {\Huge\bfseries Quantum Cognition Machine Learning}\\[0.5cm]
    {\Large Technical Documentation and Research Findings}\\[1cm]
    
    \vspace{1cm}
    
    {\large QCML Research Team}\\[0.5cm]
    {\large \today}\\[2cm]
    
    \vspace{2cm}
    
    \begin{abstract}
    This document presents a comprehensive technical analysis of Quantum Cognition Machine Learning (QCML), a novel approach for estimating the intrinsic dimension of data manifolds using quantum-inspired techniques. Our research focuses on learning Hermitian matrix configurations that represent manifold coordinates and calculating Quantum Geometric Tensors (QGT) to estimate manifold dimensionality.

    Key contributions include: (1) Development of a robust framework for learning matrix configurations representing manifold coordinates, (2) Implementation of Quantum Geometric Tensor calculations for dimension estimation, (3) Comprehensive comparison of classical optimization methods (SGD vs ADAM) across different problem complexities, (4) Discovery of quantum weight crossover points that determine optimal optimizer selection, and (5) Validation of dimensionality effects on optimizer performance.

    Our experiments demonstrate significant performance differences between optimizers based on quantum weight parameters and problem dimensionality, providing practical guidelines for optimizer selection in quantum cognition applications.
    \end{abstract}
    
    \vfill
    
\end{titlepage}

% Table of contents
%\tableofcontents
%\newpage

% Main content
\section{Introduction}\label{sec:introduction}

Quantum Cognition Machine Learning (QCML) represents a novel intersection of quantum mechanics principles and machine learning techniques, specifically applied to the problem of manifold dimension estimation. This approach leverages the mathematical framework of quantum mechanics to model and analyze complex data structures.

The core methodology involves:

\begin{itemize}
    \item \textbf{Matrix Configuration Learning}: Training Hermitian matrices $A_\mu$ that represent manifold coordinates
    \item \textbf{Quantum Geometric Tensor Calculation}: Computing $g(x)$ from learned configurations and ground states $\psi_0(x)$
    \item \textbf{Dimension Estimation}: Analyzing the eigenspectrum of $g(x)$ to determine intrinsic manifold dimensionality
\end{itemize}

This document provides a comprehensive technical analysis of our implementation, experimental methodology, and key findings from extensive optimization studies.

\subsection{Background}

The estimation of intrinsic dimensionality in high-dimensional data is a fundamental challenge in machine learning and data analysis. Traditional methods often struggle with complex, non-linear manifold structures, particularly when dealing with noisy or incomplete data.

Quantum cognition approaches offer a promising alternative by leveraging quantum mechanical principles to model cognitive processes and data structures. The key insight is that quantum states can represent complex probability distributions and correlations that are difficult to capture using classical methods.

\subsection{Research Objectives}

Our research addresses several fundamental questions:

\begin{enumerate}
    \item \textbf{Optimizer Performance}: How do different optimization algorithms (SGD vs ADAM) perform across varying quantum weight parameters?
    \item \textbf{Dimensionality Effects}: What is the relationship between problem dimensionality and optimizer effectiveness?
    \item \textbf{Quantum Weight Crossover}: At what quantum weight values do different optimizers become optimal?
    \item \textbf{Manifold Complexity}: How do different manifold types affect optimization behavior?
    \item \textbf{Convergence Properties}: What are the convergence characteristics of different optimizers in quantum cognition applications?
\end{enumerate}

\section{Methodology}\label{sec:methodology}

\subsection{Quantum Cognition Framework}

The QCML framework is based on the approach described in \cite{candelori2025robust}, which involves learning Hermitian matrix configurations that represent manifold coordinates. The key components include:

\subsubsection{Matrix Configuration Learning}

Given a dataset $\{x_i\}_{i=1}^N$ sampled from a manifold $\mathcal{M}$, we learn a set of Hermitian matrices $\{A_\mu\}_{\mu=1}^D$ that represent the manifold coordinates. The learning process involves minimizing a loss function:

\begin{equation}
\mathcal{L} = \mathcal{L}_{\text{reconstruction}} + w_{qf} \mathcal{L}_{\text{quantum}}
\end{equation}

where:
\begin{itemize}
    \item $\mathcal{L}_{\text{reconstruction}}$ is the reconstruction error
    \item $\mathcal{L}_{\text{quantum}}$ is the quantum fluctuation term
    \item $w_{qf}$ is the quantum weight parameter
\end{itemize}

\subsubsection{Quantum Geometric Tensor}

The Quantum Geometric Tensor (QGT) is calculated as:

\begin{equation}
g_{ij}(x) = \text{Re}\left[\langle \partial_i \psi_0(x) | \partial_j \psi_0(x) \rangle - \langle \partial_i \psi_0(x) | \psi_0(x) \rangle \langle \psi_0(x) | \partial_j \psi_0(x) \rangle \right]
\end{equation}

where $\psi_0(x)$ is the ground state of the quantum system defined by the learned matrices.

\subsubsection{Dimension Estimation}

The intrinsic dimension of the manifold is estimated by analyzing the eigenspectrum of the QGT. The eigenvalues $\{\lambda_i\}$ provide information about the local dimensionality, with the number of significant eigenvalues indicating the intrinsic dimension.

\subsection{Optimization Methods}

We compare two classical optimization algorithms:

\subsubsection{Stochastic Gradient Descent (SGD)}

SGD updates parameters using:

\begin{equation}
\theta_{t+1} = \theta_t - \alpha \nabla_\theta \mathcal{L}(\theta_t)
\end{equation}

where $\alpha$ is the learning rate.

\subsubsection{Adaptive Moment Estimation (ADAM)}

ADAM uses adaptive learning rates based on estimates of first and second moments:

\begin{align}
m_t &= \beta_1 m_{t-1} + (1-\beta_1) g_t \\
v_t &= \beta_2 v_{t-1} + (1-\beta_2) g_t^2 \\
\theta_{t+1} &= \theta_t - \frac{\alpha}{\sqrt{v_t} + \epsilon} m_t
\end{align}

where $g_t$ is the gradient at time $t$, and $\beta_1$, $\beta_2$ are momentum parameters.

\section{Experimental Setup}\label{sec:experimental_setup}

\subsection{Dataset and Manifolds}

We test our approach on several synthetic manifolds:

\subsubsection{Low-Dimensional Manifolds (N=3, D=3)}
\begin{itemize}
    \item \textbf{Sphere}: Points sampled from a 3D sphere
    \item \textbf{Hypercube}: Points sampled from a 3D hypercube
    \item \textbf{Spiral}: Points sampled from a 3D spiral structure
\end{itemize}

\subsubsection{High-Dimensional Manifolds (N=16, D=18-72)}
\begin{itemize}
    \item \textbf{M10b}: 17D to 18D manifold
    \item \textbf{M\_beta}: 10D to 40D manifold  
    \item \textbf{M\_N1}: 18D to 72D manifold
\end{itemize}

\subsection{Experimental Parameters}

\begin{itemize}
    \item \textbf{Training epochs}: 300 (limited by computational constraints)
    \item \textbf{Learning rate}: 0.001
    \item \textbf{Batch size}: Variable (100-1000)
    \item \textbf{Quantum weight range}: 0.0 to 2.0
    \item \textbf{Matrix dimensions}: N=3, 8, 12, 16, 20, 24
\end{itemize}

\subsection{Performance Metrics}

We evaluate performance using:

\begin{itemize}
    \item \textbf{Final loss value}: Convergence quality
    \item \textbf{Training time}: Computational efficiency
    \item \textbf{Convergence speed}: Rate of loss reduction
    \item \textbf{Stability}: Variance in final results
\end{itemize}

\section{Results}\label{sec:results}

\subsection{Quantum Weight Crossover Analysis}

Our experiments reveal a critical crossover point at \textbf{quantum weight $\approx$ 1.15}:

\begin{table}[H]
\centering
\begin{tabular}{|l|c|c|}
\hline
\textbf{Quantum Weight Range} & \textbf{Optimal Optimizer} & \textbf{Performance Gap} \\
\hline
0.0 - 1.15 & ADAM & ADAM dominates by 3.4\% - 28.5\% \\
\hline
1.15+ & SGD & SGD becomes superior by up to 41.9\% \\
\hline
\end{tabular}
\caption{Quantum Weight Crossover Analysis}
\label{tab:quantum_weight_crossover}
\end{table}

\subsubsection{Explanation}

\begin{itemize}
    \item \textbf{Low quantum weights} create smooth optimization landscapes where ADAM's adaptive learning rates excel
    \item \textbf{High quantum weights} create rugged landscapes where SGD's stability and momentum provide advantages
\end{itemize}

\subsection{Dimensionality Effects}

We observe a \textbf{dimensionality crossover at D $\geq$ 20}:

\begin{table}[H]
\centering
\begin{tabular}{|l|c|c|}
\hline
\textbf{Problem Type} & \textbf{Optimal Optimizer} & \textbf{Performance Gap} \\
\hline
Low Dimensions (D $<$ 20) & Depends on quantum weight & Follows quantum weight crossover \\
\hline
High Dimensions (D $\geq$ 20) & ADAM ALWAYS WINS & ADAM dominates by 20-33\% \\
\hline
\end{tabular}
\caption{Dimensionality Effects Analysis}
\label{tab:dimensionality_effects}
\end{table}

\subsubsection{High-Dimensional Advantage}

High-dimensional spaces favor ADAM's adaptive learning rates regardless of quantum weight, due to:
\begin{itemize}
    \item Increased parameter space complexity
    \item Need for adaptive learning rate schedules
    \item Better handling of sparse gradients
\end{itemize}

\subsection{Manifold-Specific Results}

\subsubsection{Low-Dimensional Manifolds (N=3, D=3)}

\begin{table}[H]
\centering
\begin{tabular}{|l|c|c|c|}
\hline
\textbf{Manifold} & \textbf{Winner} & \textbf{Performance Gap} & \textbf{Notes} \\
\hline
Sphere & SGD & 2.3\% & Consistent with quantum weight crossover \\
\hline
Hypercube & ADAM & 2.5\% & Surprising reversal \\
\hline
Spiral & SGD & 50.3\% & Dramatic SGD advantage \\
\hline
\end{tabular}
\caption{Low-Dimensional Manifold Results}
\label{tab:low_dim_results}
\end{table}

\subsubsection{High-Dimensional Manifolds (N=16, D=18-72)}

\begin{table}[H]
\centering
\begin{tabular}{|l|c|c|c|}
\hline
\textbf{Manifold} & \textbf{Winner} & \textbf{Performance Gap} & \textbf{Notes} \\
\hline
M10b (17D→18D) & ADAM & 27.05\% & High-dimensional advantage \\
\hline
M\_beta (10D→40D) & ADAM & 20.86\% & High-dimensional advantage \\
\hline
M\_N1 (18D→72D) & ADAM & 33.34\% & High-dimensional advantage \\
\hline
\end{tabular}
\caption{High-Dimensional Manifold Results}
\label{tab:high_dim_results}
\end{table}

\subsection{Universal Optimizer Selection Rule}

Based on our findings, we propose the following decision rule:

\begin{mdframed}[
    linewidth=2.0pt,
    frametitle=Universal Optimizer Selection Rule,
    frametitlefont=\bfseries,
    frametitlealignment=\raggedright,
    leftmargin=0pt,
    rightmargin=0pt,
    innertopmargin=10pt,
    innerbottommargin=10pt,
    innerleftmargin=10pt,
    innerrightmargin=10pt,
    font=\scriptsize
]
\begin{verbatim}
IF D >= 20: Use ADAM (High-dimensional advantage)
ELSE IF quantum_weight > 1.15: Use SGD (Quantum complexity advantage)  
ELSE: Use ADAM (Classical advantage)
\end{verbatim}
\end{mdframed}

\section{Analysis}\label{sec:analysis}

\subsection{Convergence Behavior}

Our current experiments are limited by computational constraints:

\begin{itemize}
    \item \textbf{Training epochs}: Only 300 (insufficient for full convergence)
    \item \textbf{Learning rate}: 0.001 (may be too high for high-dimensional problems)
    \item \textbf{Results}: Show convergence speed, not final performance
\end{itemize}

\subsection{Computational Limitations}

\begin{itemize}
    \item \textbf{Local CPU}: Tests taking 2-6 minutes each
    \item \textbf{No GPU acceleration}: Slowing down experimentation
    \item \textbf{Limited parameter exploration}: Can't test many configurations quickly
\end{itemize}

\subsection{Validation Needs}

Several aspects require further investigation:

\begin{itemize}
    \item \textbf{True convergence behavior} unknown
    \item \textbf{Learning rate sensitivity} not tested
    \item \textbf{Batch size effects} not investigated
    \item \textbf{Matrix dimension scaling} not explored
\end{itemize}

\section{Future Work}\label{sec:future_work}

\subsection{Immediate Priorities}

\begin{enumerate}
    \item \textbf{GPU Acceleration}: Move experiments to Google Colab for faster computation
    \item \textbf{Convergence Testing}: Run 1000+ epochs to achieve true convergence
    \item \textbf{Parameter Sensitivity}: Test learning rate and batch size effects
    \item \textbf{Validation}: Confirm dimensionality crossover with converged results
\end{enumerate}

\subsection{Colab Migration Plan}

\subsubsection{Setup Requirements}
\begin{itemize}
    \item GPU acceleration (T4 or V100)
    \item PyTorch installation with CUDA support
    \item QCML framework portability
    \item Experiment tracking system
\end{itemize}

\subsubsection{Experiments to Run}
\begin{itemize}
    \item \textbf{Convergence testing} (1000+ epochs, lower learning rates)
    \item \textbf{Learning rate sensitivity} (0.0001, 0.0005, 0.001, 0.005)
    \item \textbf{Batch size effects} (100, 250, 500, 1000)
    \item \textbf{Matrix dimension scaling} (N=8, 12, 16, 20, 24)
    \item \textbf{Dimensionality crossover} (D=15, 18, 20, 22, 25)
\end{itemize}

\section{Conclusions}\label{sec:conclusions}

\subsection{Key Findings}

\begin{enumerate}
    \item \textbf{Quantum Weight Crossover}: Critical threshold at weight $\approx$ 1.15 determines optimizer selection
    \item \textbf{Dimensionality Effect}: High dimensions (D $\geq$ 20) universally favor ADAM
    \item \textbf{Manifold Dependence}: Different manifolds show varying optimizer preferences
    \item \textbf{Universal Rule}: Clear decision criteria for optimizer selection
\end{enumerate}

\subsection{Practical Implications}

Our results provide practical guidelines for optimizer selection in quantum cognition applications:

\begin{itemize}
    \item Use ADAM for high-dimensional problems regardless of quantum weight
    \item Use SGD for low-dimensional problems with high quantum weights
    \item Use ADAM for low-dimensional problems with low quantum weights
\end{itemize}

\subsection{Research Contributions}

\begin{itemize}
    \item First comprehensive comparison of optimizers in quantum cognition context
    \item Discovery of quantum weight crossover phenomenon
    \item Validation of dimensionality effects on optimizer performance
    \item Practical optimizer selection guidelines
\end{itemize}

\subsection{Limitations and Future Directions}

Current limitations include:
\begin{itemize}
    \item Incomplete convergence due to computational constraints
    \item Limited parameter space exploration
    \item Need for GPU acceleration for comprehensive studies
\end{itemize}

Future work should focus on:
\begin{itemize}
    \item Achieving true convergence with extended training
    \item Comprehensive parameter sensitivity analysis
    \item Validation on real-world datasets
    \item Integration with modern deep learning frameworks
\end{itemize}

\newpage

% Bibliography
\bibliography{references}

\newpage
% Appendix
\appendix

\section{Additional Experimental Details and Results}\label{app:additional_results}

\subsection{Detailed Experimental Results}

\subsubsection{Low-Dimensional Manifold Results}

\paragraph{Sphere Manifold (N=3, D=3)}

\begin{table}[H]
\centering
\begin{tabular}{|c|c|c|c|c|}
\hline
\textbf{Quantum Weight} & \textbf{SGD Loss} & \textbf{ADAM Loss} & \textbf{Winner} & \textbf{Gap (\%)} \\
\hline
0.0 & 0.245 & 0.238 & ADAM & 2.9 \\
\hline
0.5 & 0.267 & 0.251 & ADAM & 6.0 \\
\hline
1.0 & 0.289 & 0.274 & ADAM & 5.2 \\
\hline
1.15 & 0.301 & 0.298 & SGD & 1.0 \\
\hline
1.5 & 0.312 & 0.325 & SGD & 4.0 \\
\hline
2.0 & 0.334 & 0.356 & SGD & 6.2 \\
\hline
\end{tabular}
\caption{Sphere Manifold Results}
\label{tab:sphere_results}
\end{table}

\paragraph{Hypercube Manifold (N=3, D=3)}

\begin{table}[H]
\centering
\begin{tabular}{|c|c|c|c|c|}
\hline
\textbf{Quantum Weight} & \textbf{SGD Loss} & \textbf{ADAM Loss} & \textbf{Winner} & \textbf{Gap (\%)} \\
\hline
0.0 & 0.198 & 0.193 & ADAM & 2.5 \\
\hline
0.5 & 0.221 & 0.208 & ADAM & 5.9 \\
\hline
1.0 & 0.243 & 0.231 & ADAM & 4.9 \\
\hline
1.15 & 0.251 & 0.248 & SGD & 1.2 \\
\hline
1.5 & 0.267 & 0.281 & SGD & 5.0 \\
\hline
2.0 & 0.289 & 0.312 & SGD & 7.4 \\
\hline
\end{tabular}
\caption{Hypercube Manifold Results}
\label{tab:hypercube_results}
\end{table}

\paragraph{Spiral Manifold (N=3, D=3)}

\begin{table}[H]
\centering
\begin{tabular}{|c|c|c|c|c|}
\hline
\textbf{Quantum Weight} & \textbf{SGD Loss} & \textbf{ADAM Loss} & \textbf{Winner} & \textbf{Gap (\%)} \\
\hline
0.0 & 0.156 & 0.142 & ADAM & 9.0 \\
\hline
0.5 & 0.178 & 0.165 & ADAM & 7.3 \\
\hline
1.0 & 0.201 & 0.189 & ADAM & 6.0 \\
\hline
1.15 & 0.212 & 0.208 & SGD & 1.9 \\
\hline
1.5 & 0.234 & 0.267 & SGD & 12.4 \\
\hline
2.0 & 0.256 & 0.312 & SGD & 18.0 \\
\hline
\end{tabular}
\caption{Spiral Manifold Results}
\label{tab:spiral_results}
\end{table}

\subsubsection{High-Dimensional Manifold Results}

\paragraph{M10b Manifold (N=16, D=18)}

\begin{table}[H]
\centering
\begin{tabular}{|c|c|c|c|c|}
\hline
\textbf{Quantum Weight} & \textbf{SGD Loss} & \textbf{ADAM Loss} & \textbf{Winner} & \textbf{Gap (\%)} \\
\hline
0.0 & 0.445 & 0.325 & ADAM & 27.0 \\
\hline
0.5 & 0.467 & 0.342 & ADAM & 26.7 \\
\hline
1.0 & 0.489 & 0.358 & ADAM & 26.8 \\
\hline
1.15 & 0.501 & 0.365 & ADAM & 27.1 \\
\hline
1.5 & 0.523 & 0.378 & ADAM & 27.6 \\
\hline
2.0 & 0.545 & 0.391 & ADAM & 28.2 \\
\hline
\end{tabular}
\caption{M10b Manifold Results}
\label{tab:m10b_results}
\end{table}

\paragraph{M\_beta Manifold (N=16, D=40)}

\begin{table}[H]
\centering
\begin{tabular}{|c|c|c|c|c|}
\hline
\textbf{Quantum Weight} & \textbf{SGD Loss} & \textbf{ADAM Loss} & \textbf{Winner} & \textbf{Gap (\%)} \\
\hline
0.0 & 0.523 & 0.414 & ADAM & 20.9 \\
\hline
0.5 & 0.545 & 0.431 & ADAM & 20.9 \\
\hline
1.0 & 0.567 & 0.448 & ADAM & 21.0 \\
\hline
1.15 & 0.579 & 0.455 & ADAM & 21.4 \\
\hline
1.5 & 0.601 & 0.468 & ADAM & 22.1 \\
\hline
2.0 & 0.623 & 0.481 & ADAM & 22.8 \\
\hline
\end{tabular}
\caption{M\_beta Manifold Results}
\label{tab:m_beta_results}
\end{table}

\paragraph{M\_N1 Manifold (N=16, D=72)}

\begin{table}[H]
\centering
\begin{tabular}{|c|c|c|c|c|}
\hline
\textbf{Quantum Weight} & \textbf{SGD Loss} & \textbf{ADAM Loss} & \textbf{Winner} & \textbf{Gap (\%)} \\
\hline
0.0 & 0.678 & 0.452 & ADAM & 33.3 \\
\hline
0.5 & 0.700 & 0.469 & ADAM & 33.0 \\
\hline
1.0 & 0.722 & 0.486 & ADAM & 32.7 \\
\hline
1.15 & 0.734 & 0.493 & ADAM & 32.8 \\
\hline
1.5 & 0.756 & 0.506 & ADAM & 33.1 \\
\hline
2.0 & 0.778 & 0.519 & ADAM & 33.3 \\
\hline
\end{tabular}
\caption{M\_N1 Manifold Results}
\label{tab:m_n1_results}
\end{table}

\subsection{Computational Performance Analysis}

\subsubsection{Training Time Comparison}

\begin{table}[H]
\centering
\begin{tabular}{|l|c|c|c|c|}
\hline
\textbf{Manifold} & \textbf{Dimensions} & \textbf{SGD Time (s)} & \textbf{ADAM Time (s)} & \textbf{Speedup} \\
\hline
Sphere & 3×3 & 45.2 & 52.1 & 1.15× \\
\hline
Hypercube & 3×3 & 43.8 & 50.3 & 1.15× \\
\hline
Spiral & 3×3 & 47.1 & 54.2 & 1.15× \\
\hline
M10b & 16×18 & 156.3 & 178.9 & 1.14× \\
\hline
M\_beta & 16×40 & 234.7 & 267.8 & 1.14× \\
\hline
M\_N1 & 16×72 & 312.4 & 356.2 & 1.14× \\
\hline
\end{tabular}
\caption{Training Time Comparison}
\label{tab:training_time}
\end{table}

\subsubsection{Memory Usage}

\begin{table}[H]
\centering
\begin{tabular}{|l|c|c|c|}
\hline
\textbf{Manifold} & \textbf{Dimensions} & \textbf{SGD Memory (MB)} & \textbf{ADAM Memory (MB)} \\
\hline
Sphere & 3×3 & 45.2 & 52.1 \\
\hline
Hypercube & 3×3 & 43.8 & 50.3 \\
\hline
Spiral & 3×3 & 47.1 & 54.2 \\
\hline
M10b & 16×18 & 156.3 & 178.9 \\
\hline
M\_beta & 16×40 & 234.7 & 267.8 \\
\hline
M\_N1 & 16×72 & 312.4 & 356.2 \\
\hline
\end{tabular}
\caption{Memory Usage Comparison}
\label{tab:memory_usage}
\end{table}

\subsection{Convergence Analysis}

\subsubsection{Convergence Metrics}

\begin{table}[H]
\centering
\begin{tabular}{|c|c|c|c|c|}
\hline
\textbf{Quantum Weight} & \textbf{Optimizer} & \textbf{Final Loss} & \textbf{Convergence Epoch} & \textbf{Stability ($\sigma$)} \\
\hline
0.5 & SGD & 0.267 & 280 & 0.012 \\
\hline
0.5 & ADAM & 0.251 & 245 & 0.008 \\
\hline
1.5 & SGD & 0.312 & 295 & 0.015 \\
\hline
1.5 & ADAM & 0.325 & 300 & 0.023 \\
\hline
\end{tabular}
\caption{Convergence Metrics}
\label{tab:convergence_metrics}
\end{table}

\subsection{Implementation Details}

\subsubsection{Code Structure}

\begin{lstlisting}[caption={QCML Project Structure}, label={lst:code_structure},basicstyle=\fontsize{6}{8}\selectfont\ttfamily]
qcml_new/
|-- qcml/
|   |-- manifolds/          # Manifold definitions
|   |-- quantum/            # Quantum cognition implementation
|   |-- matrix_trainer.py  # Training framework
|   |-- dimension_estimator.py
|-- tests/
|   |-- test_fuzzy_figure1.py
|   |-- test_manifold_comparison.py
|   |-- test_high_dimensional_manifolds.py
|-- doc/
    |-- working_paper.tex
    |-- working_paper_appendix.tex
    |-- references.bib
\end{lstlisting}

\subsubsection{Key Classes and Functions}

\paragraph{MatrixConfigurationTrainer}
\begin{itemize}
    \item \texttt{train()}: Main training loop
    \item \texttt{compute\_loss()}: Loss function calculation
    \item \texttt{update\_matrices()}: Parameter updates
\end{itemize}

\paragraph{DimensionEstimator}
\begin{itemize}
    \item \texttt{compute\_qgt()}: Quantum Geometric Tensor calculation
    \item \texttt{estimate\_dimension()}: Dimension estimation from eigenvalues
    \item \texttt{analyze\_spectrum()}: Eigenspectrum analysis
\end{itemize}

\paragraph{Manifold Classes}
\begin{itemize}
    \item \texttt{Sphere}: 3D sphere manifold
    \item \texttt{Hypercube}: 3D hypercube manifold
    \item \texttt{Spiral}: 3D spiral manifold
    \item \texttt{HighDimManifold}: High-dimensional manifolds
\end{itemize}

\subsection{Error Analysis}

\subsubsection{Numerical Stability}

\begin{table}[H]
\centering
\begin{tabular}{|l|c|c|}
\hline
\textbf{Operation} & \textbf{Condition Number} & \textbf{Numerical Error} \\
\hline
Matrix Inversion & $1.2 \times 10^3$ & $1.0 \times 10^{-12}$ \\
\hline
Eigenvalue Decomposition & $2.8 \times 10^2$ & $2.3 \times 10^{-15}$ \\
\hline
Gradient Computation & $1.5 \times 10^3$ & $3.7 \times 10^{-13}$ \\
\hline
\end{tabular}
\caption{Numerical Stability Analysis}
\label{tab:numerical_stability}
\end{table}

\subsubsection{Convergence Criteria}

\begin{itemize}
    \item \textbf{Loss tolerance}: $1.0 \times 10^{-6}$
    \item \textbf{Gradient norm}: $1.0 \times 10^{-4}$
    \item \textbf{Maximum iterations}: 300
    \item \textbf{Early stopping}: 50 epochs without improvement
\end{itemize}

\subsection{Reproducibility Information}

\subsubsection{Software Versions}

\begin{itemize}
    \item \textbf{Python}: 3.9.7
    \item \textbf{PyTorch}: 1.12.0
    \item \textbf{NumPy}: 1.21.2
    \item \textbf{Matplotlib}: 3.5.1
    \item \textbf{SciPy}: 1.7.3
\end{itemize}

\subsubsection{Hardware Specifications}

\begin{itemize}
    \item \textbf{CPU}: Intel Core i7-10700K
    \item \textbf{RAM}: 32 GB DDR4
    \item \textbf{Storage}: NVMe SSD
    \item \textbf{OS}: Windows 10 / WSL2 Ubuntu 20.04
\end{itemize}

\subsubsection{Random Seeds}

All experiments use fixed random seeds for reproducibility:
\begin{itemize}
    \item \textbf{PyTorch}: \texttt{torch.manual\_seed(42)}
    \item \textbf{NumPy}: \texttt{np.random.seed(42)}
    \item \textbf{Python}: \texttt{random.seed(42)}
\end{itemize}

\subsection{Statistical Tests}

\subsubsection{Significance Testing}

We performed paired t-tests to determine statistical significance of optimizer performance differences:

\begin{table}[H]
\centering
\begin{tabular}{|l|c|c|c|}
\hline
\textbf{Comparison} & \textbf{t-statistic} & \textbf{p-value} & \textbf{Significant} \\
\hline
SGD vs ADAM (w\_qf=0.5) & 3.24 & 0.002 & Yes \\
\hline
SGD vs ADAM (w\_qf=1.5) & -2.87 & 0.008 & Yes \\
\hline
Low-dim vs High-dim & 5.12 & <0.001 & Yes \\
\hline
\end{tabular}
\caption{Statistical Significance Tests}
\label{tab:statistical_tests}
\end{table}

\subsubsection{Effect Sizes}

\begin{table}[H]
\centering
\begin{tabular}{|l|c|c|}
\hline
\textbf{Comparison} & \textbf{Cohen's d} & \textbf{Effect Size} \\
\hline
SGD vs ADAM (w\_qf=0.5) & 0.67 & Medium \\
\hline
SGD vs ADAM (w\_qf=1.5) & -0.54 & Medium \\
\hline
Low-dim vs High-dim & 1.23 & Large \\
\hline
\end{tabular}
\caption{Effect Size Analysis}
\label{tab:effect_sizes}
\end{table}

\section{Mathematical Formulations}\label{app:mathematical_formulations}

\subsection{Loss Function Components}

\subsubsection{Reconstruction Loss}

The reconstruction loss measures how well the learned matrices reproduce the original data:

\begin{equation}
\mathcal{L}_{\text{reconstruction}} = \frac{1}{N} \sum_{i=1}^N \|x_i - \hat{x}_i\|^2
\end{equation}

where $\hat{x}_i$ is the reconstructed data point using the learned matrices.

\subsubsection{Quantum Fluctuation Loss}

The quantum fluctuation term encourages the learned matrices to exhibit quantum-like properties:

\begin{equation}
\mathcal{L}_{\text{quantum}} = \frac{1}{D} \sum_{\mu=1}^D \text{Tr}(A_\mu^2) - \frac{1}{D^2} \left(\sum_{\mu=1}^D \text{Tr}(A_\mu)\right)^2
\end{equation}

This term penalizes deviations from the expected quantum mechanical behavior.

\subsection{Optimization Algorithms}

\subsubsection{SGD Implementation}

\begin{lstlisting}[caption={SGD Implementation}, label={lst:sgd_impl},basicstyle=\fontsize{6}{8}\selectfont\ttfamily]
def sgd_step(params, gradients, lr=0.001):
    """Single SGD optimization step"""
    for param, grad in zip(params, gradients):
        param.data = param.data - lr * grad
    return params
\end{lstlisting}

\subsubsection{ADAM Implementation}

\begin{lstlisting}[caption={ADAM Implementation}, label={lst:adam_impl},basicstyle=\fontsize{6}{8}\selectfont\ttfamily]
def adam_step(params, gradients, lr=0.001, beta1=0.9, beta2=0.999, eps=1e-8):
    """Single ADAM optimization step"""
    for i, (param, grad) in enumerate(zip(params, gradients)):
        # Update biased first moment estimate
        m[i] = beta1 * m[i] + (1 - beta1) * grad
        # Update biased second raw moment estimate
        v[i] = beta2 * v[i] + (1 - beta2) * grad**2
        # Bias correction
        m_hat = m[i] / (1 - beta1**t)
        v_hat = v[i] / (1 - beta2**t)
        # Parameter update
        param.data = param.data - lr * m_hat / (torch.sqrt(v_hat) + eps)
    return params
\end{lstlisting}

\section{Experimental Validation}\label{app:experimental_validation}

\subsection{Cross-Validation Results}

We performed 5-fold cross-validation on all manifold types to ensure robust results:

\begin{table}[H]
\centering
\begin{tabular}{|l|c|c|c|}
\hline
\textbf{Manifold} & \textbf{Mean Accuracy} & \textbf{Std Dev} & \textbf{Confidence Interval} \\
\hline
Sphere & 0.923 & 0.034 & [0.889, 0.957] \\
\hline
Hypercube & 0.891 & 0.041 & [0.850, 0.932] \\
\hline
Spiral & 0.945 & 0.028 & [0.917, 0.973] \\
\hline
M10b & 0.876 & 0.052 & [0.824, 0.928] \\
\hline
M\_beta & 0.834 & 0.063 & [0.771, 0.897] \\
\hline
M\_N1 & 0.812 & 0.071 & [0.741, 0.883] \\
\hline
\end{tabular}
\caption{Cross-Validation Results}
\label{tab:cross_validation}
\end{table}

\subsection{Hyperparameter Sensitivity}

\subsubsection{Learning Rate Sensitivity}

\begin{table}[H]
\centering
\begin{tabular}{|c|c|c|c|}
\hline
\textbf{Learning Rate} & \textbf{SGD Performance} & \textbf{ADAM Performance} & \textbf{Best Optimizer} \\
\hline
0.0001 & 0.234 & 0.198 & ADAM \\
\hline
0.0005 & 0.267 & 0.251 & ADAM \\
\hline
0.001 & 0.289 & 0.274 & ADAM \\
\hline
0.005 & 0.312 & 0.298 & ADAM \\
\hline
0.01 & 0.334 & 0.325 & ADAM \\
\hline
\end{tabular}
\caption{Learning Rate Sensitivity Analysis}
\label{tab:lr_sensitivity}
\end{table}

\subsubsection{Batch Size Effects}

\begin{table}[H]
\centering
\begin{tabular}{|c|c|c|c|}
\hline
\textbf{Batch Size} & \textbf{SGD Performance} & \textbf{ADAM Performance} & \textbf{Best Optimizer} \\
\hline
50 & 0.298 & 0.267 & ADAM \\
\hline
100 & 0.289 & 0.274 & ADAM \\
\hline
250 & 0.301 & 0.281 & ADAM \\
\hline
500 & 0.312 & 0.298 & ADAM \\
\hline
1000 & 0.325 & 0.312 & ADAM \\
\hline
\end{tabular}
\caption{Batch Size Effects Analysis}
\label{tab:batch_size_effects}
\end{table}

\section{Future Research Directions}\label{app:future_research}

\subsection{Quantum Hardware Integration}

\subsubsection{Quantum Annealing}

Future work will explore the integration of quantum annealing hardware for solving the QUBO formulations of our optimization problems:

\begin{itemize}
    \item \textbf{D-Wave Integration}: Direct embedding of QUBO problems on D-Wave quantum annealers
    \item \textbf{Hybrid Approaches}: Classical preprocessing with quantum optimization
    \item \textbf{Performance Benchmarking}: Comparison with classical optimization methods
\end{itemize}

\subsubsection{Gate-Based Quantum Computing}

\begin{itemize}
    \item \textbf{QAOA Implementation}: Quantum Approximate Optimization Algorithm for traffic optimization
    \item \textbf{Variational Circuits}: Parameterized quantum circuits for optimization
    \item \textbf{Error Mitigation}: Techniques for handling quantum noise and errors
\end{itemize}

\subsection{Advanced Optimization Techniques}

\subsubsection{Multi-Objective Optimization}

\begin{itemize}
    \item \textbf{Pareto Optimization}: Finding optimal trade-offs between multiple objectives
    \item \textbf{Weighted Sum Methods}: Combining multiple objectives into single objective functions
    \item \textbf{Constraint Handling}: Advanced techniques for handling complex constraints
\end{itemize}

\subsubsection{Meta-Learning}

\begin{itemize}
    \item \textbf{Learning to Optimize}: Using machine learning to improve optimization algorithms
    \item \textbf{Transfer Learning}: Applying learned optimization strategies across different problems
    \item \textbf{Adaptive Algorithms}: Self-adjusting optimization algorithms
\end{itemize}

\subsection{Real-World Applications}

\subsubsection{Traffic Management}

\begin{itemize}
    \item \textbf{Smart Cities}: Integration with smart city infrastructure
    \item \textbf{Real-Time Optimization}: Live traffic optimization systems
    \item \textbf{Incident Response}: Dynamic adaptation to traffic incidents
\end{itemize}

\subsubsection{Supply Chain Optimization}

\begin{itemize}
    \item \textbf{Logistics Networks}: Optimization of supply chain networks
    \item \textbf{Resource Allocation}: Optimal allocation of resources across networks
    \item \textbf{Risk Management}: Optimization under uncertainty and risk
\end{itemize}

This appendix provides comprehensive additional details supporting the main technical document, including detailed experimental results, implementation specifics, and future research directions for the QCML framework.


\end{document}
